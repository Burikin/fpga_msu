
Вывод на консоль во время симуляции. 

В большинстве языков программирования существует стандартный вывод - механизм, обеспечивающий отображение текстовой информации на мониторе во время работы программы. Язык VHDL также предоставляет такую возможность. Но эта возможность не относится к синтезируемому подмножеству языка - текстовый вывод можно осуществлять только из тестбенча во время моделирования разрабатываемой схемы. В качестве консоли будет выступать текстовое окно вывода среды, в которой происходит моделирование, например Vivado. Хотя среда моделирования предоставляет возможность непосредственно наблюдать значения всех сигналов проекта в окне времянных диаграмм, часто бывает полезно выводит и текстовые сообщения. Эта функциональность обеспечивается стандартной библиотекой textio. Для того чтобы воспользоваться ее функциями необходимо добавить выражение

use textio.all;

непосредственно перед каждой архитектурой, использующей вывод на консоль. На самом деле, с помощью этой библиотеки можно также и осуществлять файловый ввод/вывод, но об этом в следующем параграфе. 

Текстовая информация может быть выдана через переменную (variable) типа line. Вывод на консоль может быть осуществлен только изнутри процесса, где все операторы исполняются последовательно. Для вывода информации в консоль, сначала ее необходимо поместить в переменную типа line, а затем вызвать специальную функцию для собственно выдачи. Следующий пример это иллюстрирует.

use textio.all;
architecture behavior of check is
begin
  process (x)
    variable s : line;
    variable cnt : integer:=0;
  begin
    if (x='1' and x'last_value='0') then
      cnt:=cnt+1;
      if (cnt>MAX_COUNT) then
        write(s,"Counter overflow - ");
        write(s,cnt);
        writeline(output,s);
      end if;
    end if;
  end process;
end behavior;


Функция write используется для добавления текстовой информации в конец переменной типа line, которая инициализируется пустой строчкой. Функция write имеет два аргумента, первый - это имя переменной, куда надо добавить данные, второй - это сами данные. В нашем примере сначала в переменную s записывается строка "Counter overflow - ", а затем текущее значение счетчика cnt конвертируется в строковое представление и добавляется в конец строчки s. После этого вызывается функция writeline, которая копирует значение строчки s в стандартный вывод среды модилирования (о чем ей сообщает зарезервированное слово output, переданное в качестве первого аргумента), после чего обнуляет эту строчку. К примеру, если значение MAX_COUNT равнялось бы 15, а сигнал x имел бы больше 15 передних фронтов, то в консоле во время моделирования появилось бы сообщение 

Counter overflow - 16

Функция write может быть использована для записи в переменную типа line следующих типов данных: bit, bit_vector, time, integer и real.



Файловый ввод/вывод

Очень часто бывает необходимо промоделировать разрабатываемую схему DUT на больщом объеме входных данных. В таких случаях входные данные храняться в файлах на диске, а моделирующая тестбенч читает их и подает на вход схемы DUT. Если в добавок тестбенч самопроверяющаяся, то и выходные данный схемы DUT необходимо сравнивать с известными опроными результатами, согласованными с входными воздействиями. В этом случае также нельзя обойтись без чтения данных из файлов.

Запись в файл при моделировании используется нечасто, хотя такая возможность есть. 

Итак, 