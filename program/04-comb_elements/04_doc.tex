\chapter{Описание комбинационных устройств на VHDL}

\emph{Последовательные и параллельные операторы. Комбинационные логические элементы и их описание на языке VHDL. Операторы выборочного и условного назначения сигналов. Оператор процесса. Запуск процесса и список чувствительности. Операторы IF и CASE. Правила кодирования комбинационных логических устройств на VHDL.}

\section{Параллельное назначение сигналов в VHDL}

В общем случае цифровые устройства можно разделить на комбинационные и последовательные. Комбинационная цепь не имеет памяти, и значения на ее выходах в каждый момент времени зависят только от значений на ее входах.  Последовательная цепь, напротив, имеет память (или внутреннее состояние), и значения на ее выходах зависят как от входных сигналов, так и от внутреннего состояния, т.е. от предыстории функционирования цепи. Операторы параллельного назначения сигналов описывают комбинационные цепи.

\subsection{Простое назначение сигналов}
\begin{Code}
\begin{lstlisting}
status       <= '1';
even         <= (p1 and p2) or (p3 and p4);
arith_result <= a + b + c - 1;
\end{lstlisting}
\end{Code}
Последнее выражение может быть изображено в виде следующей схемы:

\begin{figure}[ht]
\centering
\begin{tikzpicture}[>=latex']
\tikzstyle{arith_op} = [draw, fill=blue!20, circle, minimum size=2em]

% inputs
\node at (0,3) (input_a) {\texttt{a}};
\node at (0,2) (input_b) {\texttt{b}};
\node at (1,1) (input_c) {\texttt{c}};
\node at (2,0) (input_1) {\texttt{1}};

% operations
\node[arith_op] at (1,2) (block_plus_1) {$+$};
\node[arith_op] at (2,1) (block_plus_2) {$+$};
\node[arith_op] at (3,0) (block_minus) {$-$};

% connections
\draw[->] (input_a) -- (1,3) -- node {} (block_plus_1);
\draw[->] (input_b) -- node {} (block_plus_1);
\draw[->] (block_plus_1) -- (2,2) -- node {} (block_plus_2);
\draw[->] (input_c) -- node {} (block_plus_2);
\draw[->] (block_plus_2) -- (3,1) -- node {} (block_minus);
\draw[->] (input_1) -- node {} (block_minus);
\draw[->] (block_minus) -- (4,0);
\end{tikzpicture}
\caption{Неоптимизированная комбинационная схема.}
\label{comb_circuit_noopt}
\end{figure}

\exsect{Упражнение.} Предположим, что задержки сумматора и блока, вычисляющего разность входных сигналов, одинаковы и равны $\delta$, предположим, что других задержек нет (например, мы не учитываем задержки в проводниках хотя они порой очень существенны). Тогда изменение входного сигнала a отразится на выходном сигнале \lstinline?arith_result? только через $3 \delta$, ведь сигнал должен пройти через три блока, каждый из которых вносит задержку $\delta$. С другой стороны изменение входного сигнала c отразится на выходном сигнале \lstinline?arith_result? только через $2 \delta$. Т.е. схема еще и несимметричная. Можно ли оптимизировать схему, чтобы временная задержка стала меньше, и схема стала симметричной? (Ответ приведен далее, но все же подумайте, попробуйте переставить блоки.)

\exsect{Ответ.} Воспользуемся коммутативностью и ассоциативностью сложения: 
\[
a + b + c - 1 = (a + b) + (c - 1).
\]

\begin{figure}[ht]
\centering
\begin{tikzpicture}[>=latex']
\tikzstyle{arith_op} = [draw, fill=blue!20, circle, minimum size=2em]

% inputs
\node at (0,4) (input_a) {\texttt{a}};
\node at (0,3) (input_b) {\texttt{b}};
\node at (0,1) (input_c) {\texttt{c}};
\node at (0,0) (input_1) {\texttt{1}};

% operations
\node[arith_op] at (1,3) (block_plus_1) {$+$};
\node[arith_op] at (1,1) (block_plus_2) {$+$};
\node[arith_op] at (2,2) (block_minus) {$-$};

% connections
\draw[->] (input_a) -- (1,4) -- node {} (block_plus_1);
\draw[->] (input_b) -- node {} (block_plus_1);
\draw[->] (input_c) -- node {} (block_plus_2);
\draw[->] (input_1) -- (1,0) -- node {} (block_plus_2);
\draw[->] (block_plus_1) -- (2,3) -- node {} (block_minus);
\draw[->] (block_plus_2) -- (2,1) -- node {} (block_minus);
\draw[->] (block_minus) -- (3,2);
\end{tikzpicture}
\caption{Оптимизированная комбинационная схема.}
\label{comb_circuit_opt}
\end{figure}

При создании комбинационных схем \textbf{нельзя} назначать сигналы с обратной связью. С синтаксической точки зрения, следующие выражения корректны, они даже могут быть использованы в определенном контексте. Но их нельзя использовать непосредственно в архитектурном теле.

\begin{Code}
\begin{lstlisting}
architecture wrong_arch of wrong_entity is
begin
    -- внутреннее состояние:
    q <= (q and (not  en)) or (d and en);
    -- осцилляции q при en = '0':
    q <= (not q and (not en)) or (d and en);
end wrong_arch
\end{lstlisting}
\end{Code}
С другой стороны, следующий код корректен.
\begin{Code}
\begin{lstlisting}
architecture right_arch of right_entity is
    signal q : std_logic_vector(3 downto 0);
begin
    q <= ('0' & q(3 downto 1)) and d;
end right_arch;
\end{lstlisting}
\end{Code}

Этот код – более компактная векторная запись следующего кода.

\begin{Code}
\begin{lstlisting}
architecture right_arch of right_entity is
    signal q : std_logic_vector(3 downto 0);
begin
    q(3) <= '0'  and d(3);
    q(2) <= q(3) and d(2);
    q(1) <= q(2) and d(1);
    q(0) <= q(1) and d(0);
    ...
end right_arch;
\end{lstlisting}
\end{Code}

\subsection{Оператор условного назначения сигнала WHEN}
Это параллельный оператор, реализующий выбор одного из возможных вариантов в зависимости от некоторых управляющих сигналов. В общем виде его можно записать следующим образом:
\begin{Code}
\begin{lstlisting}
 сигнал <= выражение_1     when условие_1     else 
          выражение_2     when условие_2     else 
          ...
          выражение_(n-1) when условие_(n-1) else 
          выражение_n;
\end{lstlisting}
\end{Code}
\exsect{Пример 1.} Восьмибитовый мультиплексор 4-в-1.
Рассмотрим комбинационную схему, описываемую таблицей истинности:
\begin{table}[h]
\centering
\begin{tabular}{|c|c|}
\hline
input       & output \\ \hline
s           & x      \\ \hline
\texttt{00} & a      \\
\texttt{01} & b      \\
\texttt{10} & c      \\
\texttt{11} & d      \\
\hline
\end{tabular}
\end{table}

VHDL код, реализующий эту схему такой:
\begin{Code}
\begin{lstlisting}
library ieee;
use ieee.std_logic_1164.all;

entity mux4 is
    port(
        a, b, c, d : in std_logic_vector(7 downto 0);
        s : in  std_logic_vector(1 downto 0);
        x : out std_logic_vector(7 downto 0)
    );
end mux4;

architecture cond_arch of mux4 is
begin
    x <= a when (s = "00") else
         b when (s = "01") else
         c when (s = "10") else
         d;
end cond_arch;
\end{lstlisting}
\end{Code}

Сигнал \lstinline?x? – это выходной информационный сигнал; \lstinline?a?, \lstinline?b?, \lstinline?c?, \lstinline?d? – входные информационные сигналы, а \lstinline?s? – это сигнал управления, на основании которого осуществляется выбор одного из вариантов. 
Рассмотрим еще пару примеров.

\exsect{Пример 2.} Бинарный дешифратор.
\begin{table}[h]
\centering
\begin{tabular}{|c|c|}
\hline
input       & output        \\ \hline
s           & x             \\ \hline
\texttt{00} & \texttt{0001} \\
\texttt{01} & \texttt{0010} \\
\texttt{10} & \texttt{0100} \\
\texttt{11} & \texttt{1000} \\
\hline
\end{tabular}
\end{table}

\begin{Code}
\begin{lstlisting}
library ieee;
use ieee.std_logic_1164.all;

entity decoder4 is
    port(
        s : in  std_logic_vector(1 downto 0);
        x : out std_logic_vector(3 downto 0)
    );
end decoder4;

architecture cond_arch of decoder4 is
begin
    x <= "0001" when (s = "00") else
         "0010" when (s = "01") else
         "0100" when (s = "10") else
         "1000";
end cond_arch;
\end{lstlisting}
\end{Code}

\exsect{Пример 3.} Шифратор приоритета.
\begin{table}[h]
\centering
\begin{tabular}{|c|c|c|}
\hline
input               & \multicolumn{2}{c|}{output} \\ \hline
r                   & code        & active         \\ \hline
\texttt{1{-}{-}{-}} & \texttt{11} & \texttt{1}     \\
\texttt{01{-}{-}}   & \texttt{10} & \texttt{1}     \\
\texttt{001-}       & \texttt{01} & \texttt{1}     \\
\texttt{0001}       & \texttt{00} & \texttt{1}     \\
\texttt{0000}       & \texttt{00} & \texttt{0}     \\
\hline
\end{tabular}
\end{table}

\begin{Code}
\begin{lstlisting}
library ieee;
use ieee.std_logic_1164.all;

entity prio_encoder42 is
    port(
        r      : in  std_logic_vector(3 downto 0);
        code   : out std_logic_vector(1 downto 0);
        active : out std_logic
    );
end prio_encoder42;

architecture cond_arch of prio_encoder42 is
begin
    code <= "11" when (r(3) = '1') else
            "10" when (r(2) = '1') else
            "01" when (r(1) = '1') else
            "00";
    active <= r(3) or r(2) or r(1) or r(0);
end cond_arch;
\end{lstlisting}
\end{Code}

Здесь выходной сигнал \lstinline?code? определяет, на какой из линий \lstinline?r(i)? присутствует логическая единица. Причем линия \lstinline?r(3)? имеет \textbf{приоритет} – если на ней будет \lstinline?'1'?, то \lstinline?code = "11"? вне зависимости от других линий. 

Поговорим о том, как реализуется оператор условного назначения сигнала в железе. Семантика оператора условного назначения сигнала подразумевает создание приоритетной схемы, т.е. условия, стоящие выше имеют больший приоритет. При синтезе этого оператора реализуются три схемы:
\begin{itemize}
\item схема, вычисляющая возможные значения выходного сигнала,
\item схема, вычисляющая условия,
\item схема определения приоритета.
\end{itemize}
Чтобы лучше разобраться, что такое схема приоритета введем понятие абстрактного мультиплексора 2-в-1.

Здесь \texttt{i1}, \texttt{i2}, \texttt{o} – $n$-битные шины, а сигнал \texttt{sel} однобитный. В зависимости от него на выход \texttt{o} идет либо \texttt{i1}, либо \texttt{i0}.

\begin{figure}[ht]
\centering
\begin{tikzpicture}[>=latex']
% inputs
\node at (0,2) (input_i1) {\texttt{i1}};
\node at (0,1) (input_i0) {\texttt{i0}};
\node at (0,0) (input_sel) {\texttt{sel}};

% mux
\draw [-,fill=blue!20](1,0.5) -- (1.5,1) -- (1.5,2) -- (1,2.5) -- cycle;

\node[right] at (1,2) (true) {\texttt{T}};
\node[right] at (1,1) (false) {\texttt{F}};

\path[->] (input_i1) edge node [font=\tiny] {/} node [above] {$n$} (true);
\path[->] (input_i0) edge node [font=\tiny] {/} node [above] {$n$} (false);

\node[right] at (2.5,1.5) (out) {\texttt{o}};
\path[->] (1.5,1.5) edge node [font=\tiny] {/} node [above] {$n$} (out);

\draw[->] (input_sel) -- (1.25,0) -- (1.25,0.75);
\end{tikzpicture}
\caption{Символ абстрактного мультиплексора 2-в-1.}
\label{abstract_mux_2_1}
\end{figure}

Когда мы пишем на VHDL выражение:

\begin{Code}
\begin{lstlisting}
signal_name <= value_expr_1 when boolean_expr_1 else 
               value_expr_2;
\end{lstlisting}
\end{Code}

реализуется следующая схема с мультиплексором:

\begin{figure}[ht]
\centering
\begin{tikzpicture}[>=latex']
\tikzstyle{expr} = [draw, fill=yellow!20, dotted, rectangle, minimum size=2em]

% inputs
\node[expr] at (-1,2) (input_i1) {\texttt{value\_expr\_1}};
\node[expr] at (-1,1) (input_i0) {\texttt{value\_expr\_2}};
\node[expr] at (-1,0) (input_sel) {\texttt{boolean\_expr\_1}};

% mux
\draw [-,fill=blue!20](1,0.5) -- (1.5,1) -- (1.5,2) -- (1,2.5) -- cycle;

\node[right] at (1,2) (true) {\texttt{T}};
\node[right] at (1,1) (false) {\texttt{F}};

\draw[->] (input_i1) -- (true);
\draw[->] (input_i0) -- (false);

\draw[->] (1.5,1.5) -- (2.5,1.5);
\node[right] at (2.5,1.5) (true) {\texttt{sig}};

\draw[->] (input_sel) -- (1.25,0) -- (1.25,0.75);
\end{tikzpicture}
\caption{Схема с мультиплексором 2-в-1.}
\label{mux_fig_1}
\end{figure}

Каждый следующий уровень ветвления оператора условного назначения сигнала добавляет еще одну ступень в иерархию вместе с еще одним мультиплексором.

\begin{Code}
\begin{lstlisting}
signal_name <= value_expr_1 when boolean_expr_1 else 
               value_expr_2 when boolean_expr_2 else 
               value_expr_3;
\end{lstlisting}
\end{Code}

\begin{figure}[ht]
\centering
\begin{tikzpicture}[>=latex']
\tikzstyle{expr} = [draw, fill=yellow!20, dotted, rectangle, minimum size=2em]

\begin{scope}
    % inputs
    \node[expr] at (-1,2) (input_i1) {\texttt{value\_expr\_1}};
    \node[expr,fill=red!20] at (-1,1) (input_rem) {\texttt{remaining\_part}};
    \node[expr] at (-1,0) (input_sel) {\texttt{boolean\_expr\_1}};

    % mux
    \draw [-,fill=blue!20](1,0.5) -- (1.5,1) -- (1.5,2) -- (1,2.5) -- cycle;

    \node[right] at (1,2) (true) {\texttt{T}};
    \node[right] at (1,1) (false) {\texttt{F}};
    
    \draw[->] (input_i1) -- (true);
    \draw[->] (input_rem) -- (false);

    \draw[->] (1.5,1.5) -- (2,1.5);
    \node[right] at (2,1.5) (true) {\texttt{sig}};

    \draw[->] (input_sel) -- (1.25,0) -- (1.25,0.75);
\end{scope}

\begin{scope}[xshift=7cm]
    % inputs
    \node[expr] at (-2.5,2) (input_i1) {\texttt{value\_expr\_1}};
    \node[expr] at (-2.5,1) (input_i2) {\texttt{value\_expr\_2}};
    \node[expr] at (-2.5,0) (input_i3) {\texttt{value\_expr\_3}};
    \node[expr] at (-2.5,-1) (input_sel2) {\texttt{boolean\_expr\_2}};
    \node[expr] at (-2.5,-2) (input_sel1) {\texttt{boolean\_expr\_1}};

    % mux
    \draw [-,fill=blue!20](1,0.5) -- (1.5,1) -- (1.5,2) -- (1,2.5) -- cycle;
    \draw [-,fill=blue!20](-0.5,-0.5) -- (0,0) -- (0,1) -- (-0.5,1.5) -- cycle;

    \node[right] at (1,2) (true1) {\texttt{T}};
    \node[right] at (1,1) (false1) {\texttt{F}};
    
    \node[right] at (-0.5,1) (true2) {\texttt{T}};
    \node[right] at (-0.5,0) (false2) {\texttt{F}};
    
    \draw[->] (input_i1) -- (true1);
    
    \draw[->] (0,0.5) -- (0.5,0.5) -- (0.5,1) -- (false1);
    
    \draw[->] (input_i2) -- (true2);
    \draw[->] (input_i3) -- (false2);

    \draw[->] (1.5,1.5) -- (2,1.5);
    \node[right] at (2,1.5) (true1) {\texttt{sig}};

    \draw[->] (input_sel2) -- (-0.25,-1) -- (-0.25,-0.25);
    \draw[->] (input_sel1) -- (1.25,-2) -- (1.25,0.75);
\end{scope}
\end{tikzpicture}
\caption{Схема с мультиплексорами 2-в-1.}
\label{mux_fig_1}
\end{figure}

\subsection{Оператор выборочного назначения сигнала WITH ... SELECT}

Это параллельный оператор, также реализующий выбор одного из возможных вариантов в зависимости от некоторых управляющих сигналов. Во многом он похож на оператор условного назначения, но есть и отличия. Записать в общем виде его можно так:

\begin{Code}
\begin{lstlisting}[mathescape]
with задающее$\_$выражение select
    сигнал <=
        выражение_1 when выбор_1,
        выражение_2 when выбор_2,
        выражение_3 when выбор_3,
        ...
        выражение_$n$ when выбор_$n$;
\end{lstlisting}
\end{Code}

Обратите внимание на использование \emph{запятых} после каждого выборочного значения (\lstinline[mathescape]?выбор_$i$?) и на точку с запятой в конце.

Выборочные значения (\lstinline[mathescape]?выбор_$i$?) должны быть \textbf{взаимно исключающими} (значения не должны перекрываться или повторяться) и \textbf{полными} (все значения должны быть перечислены). Для обеспечения полноты выборочных значений рекомендуется использовать \lstinline?others?. Заметим, что выборочные значения~-- это именно значения, а не условия.

Примеры из предыдущего параграфа можно переписать с использование оператора выборочного назначения сигнала.

\exsect{Пример 1.} Мультиплексор 4-в-1.

\begin{Code}
\begin{lstlisting}
architecture sel_arch of mux4 is
begin
    with s select
        x <= a when "00",
             b when "01",
             c when "10",
             d when others;
end sel_arch;
\end{lstlisting}
\end{Code}

\exsect{Пример 2.} Бинарный дешифратор.

\begin{Code}
\begin{lstlisting}
architecture sel_arch of decoder4 is
begin
    with sel select
        x <= "0001" when "00",
             "0010" when "01",
             "0100" when "10",
             "1000" when others;
end sel_arch;
\end{lstlisting}
\end{Code}

\exsect{Пример 3.} Шифратор приоритета.

\begin{Code}
\begin{lstlisting}
architecture sel_arch of prio_encoder42 is
begin
    with r select
        code <= "11" when "1000" | "1001" | "1010" |
                          "1011" | "1100" | "1101" |
                          "1110" | "1111",
                "10" when "0100" | "0101" | "0110" |
                          "0111",
                "01" when "0010" | "0011",
                "00" when others;
        active <= r(3) or r(2) or r(1) or r(0);
end sel_arch;
\end{lstlisting}
\end{Code}

В последнем примере вертикальной чертой \lstinline?"|"? идет компактное перечисление вариантов, которым соответствует один выходной код.

При синтезе оператора выборочного назначения сигнала реализуется равноправный абстрактный мультиплексор $k$-в-1.

\begin{figure}[ht]
\centering
\begin{tikzpicture}[>=latex']
% inputs
\node (input_ik) at (0,3.5) {\texttt{ik}};
\node (input_i1) at (0,2) {\texttt{i1}};
\node (input_i0) at (0,1.5) {\texttt{i0}};
\node (input_sel) at (0,0) {\texttt{sel}};

% mux
\draw [-,fill=blue!20](1,0.5) -- (2,1.5) -- (2,3.5) -- (1,4.5) -- cycle;

\node[right] (ck) at (1,3.5) {\texttt{ck}};
\node[right] (c1) at (1,2) {\texttt{c1}};
\node[right] (c0) at (1,1.5) {\texttt{c0}};

\path[->] (input_ik) edge node [font=\tiny] {/} node [above] {$n$} (ck);
\path[->] (input_i1) edge node [font=\tiny] {/} node [above] {$n$} (c1);
\path[->] (input_i0) edge node [font=\tiny] {/} node [above] {$n$} (c0);

\draw[fill=black] (0.65,3.1) circle (1pt);
\draw[fill=black] (0.65,2.85) circle (1pt);
\draw[fill=black] (0.65,2.6) circle (1pt);

\node[right] (out) at (3,2.5) {\texttt{o}};
\path[->] (2,2.5) edge node [font=\tiny] {/} node [above] {$n$} (out);

\draw[->] (input_sel) -- (1.5,0) -- (1.5,1);
\end{tikzpicture}
\caption{Абстрактный мультиплексор $k$-в-1.}
\label{abstract_mux_k_1}
\end{figure}

В данной схеме \textbf{нет приоритета} при продвижении сигнала с входа на выход схемы. Также можно сказать, что схема получается симметричная с одинаковыми задержками на всех входных линиях.

Проще всего сравнить операторы выборочного и условного назначения сигналов на примере мультиплексора 4-в-1.

При использовании оператора условного назначения будет реализована следующая схема:

\begin{figure}[ht]
\centering
\begin{tikzpicture}[>=latex']
\tikzstyle{branch}=[fill,shape=circle,minimum size=3pt,inner sep=0pt]

% inputs
\node (input_A) at (-4,6.5) {\texttt{A}};
\node (input_B) at (-4,5.5) {\texttt{B}};
\node (input_C) at (-4,4.5) {\texttt{C}};
\node (input_D) at (-4,3.5) {\texttt{D}};
\node (input_S0) at (-4,1.5) {\texttt{S(0)}};
\node (input_S1) at (-4,1.0) {\texttt{S(1)}};

% mux
\draw [-,fill=blue!20](1,5) -- (1.5,5.5) -- (1.5,6.5) -- (1,7) -- cycle;
\node[right] (true_S1) at (1,6.5) {\texttt{T}};
\node[right] (false_S1) at (1,5.5) {\texttt{F}};
\node[right] (sel_S1) at (1.25,5.25) {\texttt{S1}};

\draw [-,fill=blue!20](-0.5,4) -- (0,4.5) -- (0,5.5) -- (-0.5,6) -- cycle;
\node[right] (true_S2) at (-0.5,5.5) {\texttt{T}};
\node[right] (false_S2) at (-0.5,4.5) {\texttt{F}};
\node[right] (sel_S2) at (-0.25,4.25) {\texttt{S2}};

\draw [-,fill=blue!20](-2,3) -- (-1.5,3.5) -- (-1.5,4.5) -- (-2,5) -- cycle;
\node[right] (true_S3) at (-2,4.5) {\texttt{T}};
\node[right] (false_S3) at (-2,3.5) {\texttt{F}};
\node[right] (sel_S3) at (-1.75,3.25) {\texttt{S3}};

\node[nor gate US, draw, logic gate inputs=nn] (Nor01) at (0.5,1.25) {};
\node[not gate US, draw, rotate=90] (Not0) at (-0.25,2.5) {};
\node[not gate US, draw, rotate=90] (Not1) at (-1.75,2.0) {};

\draw[->] (input_A) -- (true_S1);
\draw[->] (input_B) -- (true_S2);
\draw[->] (input_C) -- (true_S3);
\draw[->] (input_D) -- (false_S3);

\node[right] (out) at (2.5,6) {\texttt{out}};
\draw[->] (1.5,6) -- (out);
\draw[->] (0,5) -- (0.5,5) -- (0.5,5.5) -- (false_S1);
\draw[->] (-1.5,4) -- (-1,4) -- (-1,4.5) -- (false_S2);

\draw[->] (Nor01.output) -| (1.25,5.25);
\draw[->] (Not0.output) -- (-0.25,4.25);
\draw[->] (Not1.output) -- (-1.75,3.25);

\draw (-0.25,1.5) node[branch] {} -- (Not0.input);
\draw (-1.75,1.0) node[branch] {} -- (Not1.input);

\draw (input_S0) -- (-0.25,1.5) -- (-0,1.5) |- (Nor01.input 1);
\draw (input_S1) -- (-1.75,1.0) -- (-0,1.0) |- (Nor01.input 2);

\end{tikzpicture}
\caption{Оператор условного назначения.}
\label{if_hard}
\end{figure}

Видно, что схема несимметричная, с различными задержками по входным сигналам. Но при этом сигнал \texttt{A} имеет приоритет над остальными.

При использовании оператора выборочного назначения сигнала реализуется следующая схема:

\begin{figure}[ht]
\centering
\begin{tikzpicture}[>=latex']
\tikzstyle{branch}=[fill,shape=circle,minimum size=3pt,inner sep=0pt]

% inputs
\node (input_A) at (-6.5,8.5) {\texttt{A}};
\node (input_B) at (-6.5,6.5) {\texttt{B}};
\node (input_C) at (-6.5,4.5) {\texttt{C}};
\node (input_D) at (-6.5,2.5) {\texttt{D}};
\node (input_S0) at (-6.5,1.0) {\texttt{S(0)}};
\node (input_S1) at (-6.5,0.5) {\texttt{S(1)}};

% gates
\node[and gate US, draw, logic gate inputs=nnn] (And1) at (-2,8) {};
\node[and gate US, draw, logic gate inputs=nnn] (And2) at (-2,6) {};
\node[and gate US, draw, logic gate inputs=nnn] (And3) at (-2,4) {};
\node[and gate US, draw, logic gate inputs=nnn] (And4) at (-2,2) {};
\node[or gate US, draw, logic gate inputs=nnnn] (Or1) at (-0,5) {};
\node[not gate US, draw] (Not1) at (-4.0,8.0) {};
\node[not gate US, draw] (Not2) at (-3.5,7.5) {};
\node[not gate US, draw] (Not3) at (-4.0,6.0) {};
\node[not gate US, draw] (Not4) at (-4.0,3.5) {};

\draw (And1.output) -- (-1,8) |- (Or1.input 1);
\draw (And2.output) -- (-1.25,6) |- (Or1.input 2);
\draw (And3.output) -- (-1.25,4) |- (Or1.input 3);
\draw (And4.output) -- (-1,2) |- (Or1.input 4);

\draw (input_A)     -- (-2.75,8.5) |- (And1.input 1);
\draw (input_S0) -- (-5.25,1.0) -- (-5.25,8.0) -- (Not1.input);
\draw (Not1.output) -- (-2.75,8.0) -- (And1.input 2);
\draw (input_S1) -- (-4.75,0.5) -- (-4.75,7.5) -- (Not2.input);
\draw (Not2.output) -- (-2.75,7.5) |- (And1.input 3);

\draw (input_B)     -- (-2.75,6.5) |- (And2.input 1);
\draw (-5.25,6.0) node[branch] {} -- (Not3.input);
\draw (Not3.output) -- (-2.75,6.0) -- (And2.input 2);
\draw (-4.75,5.5) node[branch] {} --  (-2.75,5.5) |- (And2.input 3);

\draw (input_C)     -- (-2.75,4.5) |- (And3.input 1);
\draw (-5.25,4.0) node[branch] {} -- (-2.75,4.0) -- (And3.input 2);
\draw (-4.75,3.5) node[branch] {} -- (Not4.input);
\draw (Not4.output) -- (-2.75,3.5) |- (And3.input 3);

\draw (input_D)     -- (-2.75,2.5) |- (And4.input 1);
\draw (-5.25,2.0) node[branch] {} -- (-2.75,2.0) -- (And4.input 2);
\draw (-4.75,1.5) node[branch] {} -- (-2.75,1.5) |- (And4.input 3);

\node[right] (out) at (1.75,5) {\texttt{out}};
\draw[->] (Or1.output) -- (out);

\end{tikzpicture}
\caption{Оператор выборочного назначения.}
\label{case_hard}
\end{figure}

Здесь схема симметричная, задержки по входным сигналам приблизительно одинаковые. Все сигналы имеют одинаковый приоритет.

У обеих схем \emph{логика работы} и \emph{таблицы истинности} \textbf{одинаковые}, при этом схемные реализации разные. При работе с языком VHDL вы всегда должны пытаться мыслить на уровне логических элементов и схем, чтобы получать эффективные реализации, ведь по сути VHDL – это не более чем удобный инструмент для описания таких объектов. 

\section{Последовательное назначение сигналов в VHDL}

В языке VHDL есть средство для описания последовательных алгоритмов с помощью, так называемых, \emph{последовательных операторов}. Эти операторы должны располагаться внутри специальной конструкции VHDL, называемой \textbf{оператор процесса}. Главной целью введения последовательных операторов в язык VHDL было описание абстрактного \emph{поведения} цифровых цепей (на более высоком уровне абстракции~-- \emph{поведенческом} (\emph{behavioural})). В отличие от операторов параллельного назначения сигналов, последовательные операторы не всегда имеют однозначное соответствие в железе. Часто даже они не могут быть синтезированы вовсе. Поэтому для описания схем с помощью процессов и последовательных операторов необходимо придерживаться четких шаблонов, которые однозначно синтезируются. Рассмотрением этих шаблонов мы и займемся. Пусть вас не пугает слово "<шаблон">: несмотря на то, что определенные компоненты должны описываться по четкой схеме, у вас останется огромное пространство для творчества при определении взаимодействия этих компонентов. 

\subsection{Оператор процесса PROCESS}

Оператор процесса имеет следующий синтаксис:

\begin{Code}
\begin{lstlisting}[mathescape]
process ( список$\_$чувствительности )
    объявления;
begin
    последовательные$\_$операторы;
    последовательные$\_$операторы;
    ...
end process;
\end{lstlisting}
\end{Code}

В подразделе объявлений оператора процесса могут быть объявлены подпрограммы, типы, подтипы, константы, переменные (только в подразделе объявлений оператора процесса), атрибуты, файлы и др. Сигналы в подразделе объявлений оператора процесса объявлены быть \textbf{не могут}.

Сам оператор процесса является \textbf{параллельным оператором}. Если в архитектурном теле разместить несколько процессов, то они будут выполняться параллельно. Но внутри каждого процесса находятся операторы, выполняемые последовательно, т.е. один за другим. Список чувствительности – это список запускающих процесс сигналов. Когда один из сигналов, указанных в списке чувствительности, меняет свое значение, процесс запускается, и последовательные операторы выполняются до конца процесса. После этого процесс приостанавливает свою работу до тех пор, пока опять какой-нибудь сигнал из списка чувствительности не изменит свое значение. Таким образом, процесс может находиться в одном из двух состояний: запущенном и приостановленном. \textbf{Для комбинационной схемы \emph{все входы} должны быть в списке чувствительности.} 

\begin{Code}
\begin{lstlisting}
signal a, b, c, y: std_logic;
...
process (a, b, c) -- правильно
begin
    y <= a and b or c;
end process;
...
process (a)       -- неправильно
begin
    y <= a and b or c;
end process;
\end{lstlisting}
\end{Code}

Внутри процесса сигналу можно присвоить значение несколько раз. При этом значение сигнала не изменится до выхода из процесса, а после выхода станет равным последнему присвоенному значению. Получается, что следующие записи просто эквивалентны.

\begin{Code}
\begin{lstlisting}
process (a, b, c, d)
begin
    y <= a or c;
    y <= a and b;
    y <= c and d;
end process;
\end{lstlisting}
\end{Code}

\begin{Code}
\begin{lstlisting}
process (a, b, c, d)
begin
    y <= c and d;
end process;
\end{lstlisting}
\end{Code}

Очень важно понимать, что последнее значение присваивается сигналу не сразу, а в конце процесса. Промежуточные значения сигнала не доступны. В этом отличие сигнала от \emph{переменной}.

\begin{Code}
\begin{lstlisting}
architecture arch of main is
    signal y : std_logic;
begin
    process (a, b, c, d)
    begin
        y <= a or c;
        -- Здесь, после оператора присвоения, сигнал y 
        -- не равен a or c.
        -- Здесь могут идти последовательные операторы, 
        -- которые не используют промежуточное значение
        -- сигнала y.
        z <= y;
        -- Здесь сигналу z не присваивается значение  
        -- a or c, ему присваивается значение c and d.
        y <= a and b;
        y <= c and d;
    end process;
    -- Зато здесь, вне тела процесса, y всегда равен c and d
end arch;
\end{lstlisting}
\end{Code}

\exsect{Замечание.} Синтезатор XST достаточно умный, чтобы не реализовать в этом конкретном простом примере схему с памятью (ведь сигнала y нет в списке чувствительности процесса), но на синтезатор рекомендуется полагаться не всегда.

На самом деле, с оператором процесса вы уже сталкивались.

\exsect{Вопрос.} Где?

\exsect{Ответ.} Правильно, при описании testbench. Там был процесс stim\_proc, задающий значения входных сигналов для тестируемой схемы.

\exsect{Вопрос} Чего не было в процессе stim\_proc?

\exsect{Ответ.} Там не было списка чувствительности. Но зато там были операторы \lstinline?wait?.

\subsection{Оператор ожидания WAIT}

Процесс с операторами ожидания имеет один или более операторов wait, но не имеет списка чувствительности. Оператор \lstinline?wait? имеет в общем случае следующий вид:

\begin{Code}
\begin{lstlisting}[mathescape]
wait on список$\_$чувствительности until условие for тайм$\_$аут;
\end{lstlisting}
\end{Code}

Оператор \lstinline?wait? является причиной временного прекращения оператора процесса или процедуры. Оператор ожидания \lstinline?wait? приостанавливает процесс до момента, пока не изменится некоторый сигнал в списке чувствительности оператора \lstinline?wait?, в этот момент будет произведено вычисление условия. Фраза "<условие">~-– есть выражение типа \lstinline?boolean?. Фраза "<тайм-аут"> – есть выражение типа \lstinline?time?, оно устанавливает максимальное время ожидания, после которого процесс возобновит свое выполнение.

\exsect{Пример.}

\begin{Code}
\begin{lstlisting}
wait on a, b until (c = 0) for 50 ns;
\end{lstlisting}
\end{Code}

Этот оператор приостановит процесс до момента изменения сигнала \lstinline?a? или сигнала \lstinline?b?, после чего будет проверено выражение \lstinline?c = 0?, и если результатом проверки будет истина, процесс возобновится. Но независимо от этих условий возобновление процесса произойдет через \lstinline?50 ns?.

Допустимо записывать одно или более условий в операторе ожидания \lstinline?wait?, например:

\begin{Code}
\begin{lstlisting}
wait on a, b;       -- Процесс возобновится, когда изменится a или b.

wait until (c = 0); -- Процесс возобновится, когда c изменит свое 
                    -- значение из 1 в 0.
wait for 50 ns;     -- Процесс возобновится через 50 нс.

wait;               -- Процесс больше не возобновится, если не 
                    -- поместить такой оператор в конце процесса, 
                    -- то выполнение процесса начнется сначала. 
                    -- Такой оператор можно использовать в конце 
                    -- testbench для окончания симуляции.
\end{lstlisting}
\end{Code}

В процессе допускается использование нескольких операторов \lstinline?wait?.

\exsect{ВАЖНО!} Только несколько четко определенных конструкций с операторами ожидания \lstinline?wait? могут быть синтезированы в железе. В документации к синтезатору XST допускается применение оператора \lstinline?wait? в последовательных схемах (не комбинационных, о них вы узнаете в следующей лекции), при этом допускается только конструкция \lstinline?wait until?, причем она должна быть единственной в процессе, и у процесса не должно быть списка чувствительности.

\textbf{Настоятельно рекомендуется использовать конструкции с оператором ожидания только для \emph{СИМУЛЯЦИИ}.}

\subsection{Переменные в VHDL}

Наряду с сигналами и константами в языке VHDL есть еще один программный элемент данных – переменная. Локальные переменные могут быть объявлены только в подразделе объявлений оператора процесса \lstinline?process? и подпрограмм (функций и процедур). Объявление переменной имеет следующий синтаксис:

\begin{Code}
\begin{lstlisting}[mathescape]
variable имя$\_$переменной : тип [ := начальное$\_$значение ];
\end{lstlisting}
\end{Code}

Выражение в квадратных скобках может быть опущено.

В каждый момент времени переменная может иметь любое, но всегда \textbf{единственное} значение. В другой момент времени переменная может иметь другое значение. Значение переменной может быть модифицировано оператором присваивания значения переменной. Этот оператор имеет следующий синтаксис:

\begin{Code}
\begin{lstlisting}[mathescape]
 имя$\_$переменной := выражение;
\end{lstlisting}
\end{Code}

При этом считается, что значение переменной модифицируется оператором присваивания значения \textbf{немедленно} (т.е. без временной задержки), как только этот оператор выполнен. Поведение оператора присвоения значения переменной в VHDL похоже на поведение оператора присвоения значения переменной в обычном традиционном языке программирования (C, C++, Pascal и др.). Этим отличаются присваивания значения сигналам и переменным.

\exsect{Пример.}

\begin{Code}
\begin{lstlisting}
signal a, b, y : std_logic;
...
process (a, b)
    variable tmp : std_logic;
begin
    tmp := '0';
    tmp := tmp or a;
    tmp := tmp or b;
    y <= tmp;
end process;
\end{lstlisting}
\end{Code}

Переменная \lstinline?tmp? получает значение немедленно в каждом последовательном выражении, содержащем оператор присвоения значения переменной (\lstinline?:=?). И затем значение переменной (равное \lstinline?'0' or a or b? $\Leftrightarrow$ \lstinline?a or b?) присваивается сигналу \lstinline?y?.

В железе такой код может быть реализован следующим образом.

\begin{figure}[ht]
\centering
\begin{tikzpicture}[>=latex']
\tikzstyle{branch}=[fill,shape=circle,minimum size=3pt,inner sep=0pt]
% gates
\node[or gate US, draw, logic gate inputs=nn] (Or1) at (2.5,0.5) {};
\node[or gate US, draw, logic gate inputs=nn] (Or2) at (5,0) {};

% inputs
\node (i0) at (0,1) {\texttt{'0'}};
\path let \p1 = (Or1.input 2) in node (ia) at (0,\y1) {\texttt{a}};
\path let \p2 = (Or2.input 2) in node (ib) at (0,\y2) {\texttt{b}};

% outputs
\node (oy) at (7.5,0) {\texttt{y}};

% connections
\path[-] (i0) edge node [above] {\texttt{tmp0}} (2,1);
\draw (2,1) |- (Or1.input 1);
\draw (ia) |- (Or1.input 2);

\path[-] (Or1.output) edge node [above] {\texttt{tmp1}} (4,0.5);
\draw (4,0.5) |- (Or2.input 1);

\draw (ib) |- (Or2.input 2);
\path[-] (Or2.output) edge node [above] {\texttt{tmp2}} (oy);
\end{tikzpicture}
\caption{Реализация в железе кода с переменными.}
\label{hardware_var_0}
\end{figure}

Для реализации предыдущего процесса в железе нам потребовалось ввести новые переменные \lstinline?tmp0?, \lstinline?tmp1?, и \lstinline?tmp2?.

Для сравнения повторим предыдущий фрагмент кода, заменив переменные сигналами.

\begin{Code}
\begin{lstlisting}
signal a, b, y, tmp : std_logic;
...
process (a, b, tmp)
begin
    tmp <= '0';
    tmp <= tmp or a;
    tmp <= tmp or b;
    y   <= tmp;
end process;
\end{lstlisting}
\end{Code}

Заметим, что сигналы должны быть "<глобальными"> и должны быть объявлены вне процесса, и сигнал tmp должен быть включен в список чувствительности процесса. Этот фрагмент кода эквивалентен следующему.

\begin{Code}
\begin{lstlisting}
signal a, b, y, tmp : std_logic;
...
process (a, b, tmp)
begin
    tmp <= tmp or b;
    y <= tmp;
end process;
\end{lstlisting}
\end{Code}

Здесь мы имеем дело с комбинационной петлей и элементом ИЛИ, как показано на следующем рисунке.

\begin{figure}[ht]
\centering
\begin{tikzpicture}[>=latex']
\tikzstyle{branch}=[fill,shape=circle,minimum size=3pt,inner sep=0pt]

\node[or gate US, draw, logic gate inputs=nn] (Or1) at (0,0) {};

\path let \p1 = (Or1.input 1) in node (ib) at (-1.5,\y1) {\texttt{b}};
\node (oy) at (2,0) {\texttt{y}};

\draw (ib) -- (Or1.input 1);
\path[-] (Or1.output) edge node [above] {\texttt{tmp}} (1.25,0);
\draw[->] (1.25,0) -- (oy);
\draw (1.25,0) node[branch] {} -- (1.25,-0.5) -- (-0.75,-0.5) |- (Or1.input 2);

\end{tikzpicture}
\caption{Комбинационная петля.}
\label{comb_loop}
\end{figure}

Заметим, что переменные локальны для процесса, в котором они объявлены, и они не видны вне этого процесса.

Прежде, чем использовать переменную, необходимо задать себе вопрос, действительно ли она нужна. Неправильное использование переменных очень непросто найти и исправить. В большинстве случаев аналогичный код можно написать без использования переменных.

\subsection{Оператор IF}
Оператор \lstinline?if?~-- это один из последовательных операторов, поэтому его можно использовать только внутри оператора процесса. Синтаксис оператора \lstinline?if? следующий:

\begin{Code}
\begin{lstlisting}[mathescape]
if условие_1 then
    последовательные$\_$операторы;
elsif условие_2 then
    последовательные$\_$операторы;
elsif условие_3 then
    последовательные$\_$операторы;
...
else
    последовательные$\_$операторы;
end if;
\end{lstlisting}
\end{Code}

Обратите внимание на написание ключевого слова \lstinline?elsif? и на то, что \lstinline?end if? пишется раздельно.

Рассмотрим его применение на примерах из предыдущего раздела.

\exsect{Пример 1.} Мультиплексор 4-в-1.

\begin{Code}
\begin{lstlisting}
architecture if_arch of mux4 is
begin
    process (a, b, c, d, s)
    begin
        if (s = "00") then
            x <= a;
        elsif (s = "01") then
            x <= b;
        elsif (s = "10") then
            x <= c;
        else
            x <= d;
        end if;
    end process;
end if_arch;
\end{lstlisting}
\end{Code}

\exsect{Пример 2.} Бинарный дешифратор.

\begin{Code}
\begin{lstlisting}
architecture if_arch of decoder4 is
begin
    process (s)
    begin
        if (s = "00") then
            x <= "0001";
        elsif (s = "01") then
            x <= "0010";
        elsif (s = "10") then
            x <= "0100";
        else
            x <= "1000";
        end if;
    end process;
end if_arch;
\end{lstlisting}
\end{Code}

\exsect{Пример 3.} Шифратор приоритета.

\begin{Code}
\begin{lstlisting}
architecture if_arch of prio_encoder42 is
begin
    process (r)
    begin
        if (r(3) = '1') then
            code <= "11";
        elsif (r(2) = '1') then
            code <= "10";
        elsif (r(1) = '1') then
            code <= "01";
        else
            code <= "00";
        end if;
    end process;
    active <= r(3) or r(2) or r(1) or r(0);
end if_arch;
\end{lstlisting}
\end{Code}

Сравним оператор \lstinline?if? с оператором условного назначения сигналов. Для простых случаев эти операторы просто эквивалентны.

\begin{Code}
\begin{lstlisting}
 сигнал <= выражение_1 when условие_1 else 
          выражение_2 when условие_2 else 
          ...
          выражение_n;
\end{lstlisting}
\end{Code}

эквивалентно:

\begin{Code}
\begin{lstlisting}
process(...)
begin
    if условие_1 then
        сигнал <= выражение_1;
    elsif условие_2 then
        сигнал <= выражение_2;
    ...
    else
        сигнал <= выражение_n;
    end if;
end process; 
\end{lstlisting}
\end{Code}

Отличия:

\begin{itemize}
\item По структуре: внутри \lstinline?if? может быть еще один \lstinline?if?.
\item По пониманию: легче понять смысл.
\item В одной ветке можно назначить несколько сигналов.
\end{itemize}

\subsubsection{Проблема неполного дерева условий оператора IF}

В VHDL только одна ветка \lstinline?then? обязательна. Остальные ветки (\lstinline?elsif?, \lstinline?else?) могут быть опущены. Здесь надо быть аккуратными. Например, рассмотрим компаратор двух чисел, выход которого равен \lstinline?'1'?, когда они равны.

\begin{Code}
\begin{lstlisting}
process(a, b)
begin
    if (a = b) then
        eq <= '1';
    end if;
end process;
\end{lstlisting}
\end{Code}

Этот код синтаксически верен, но приводит к неверной реализации. Так как здесь нет ветки \lstinline?else?, то, когда числа не равны, не производится никакого действия. Семантика VHDL такова, что в этом случае значение сигнала \lstinline?eq? не обновляется, и он сохраняет свое значение. Это эквивалентно следующей записи:

\begin{Code}
\begin{lstlisting}
process(a, b)
begin
    if (a = b) then
        eq <= '1';
    else
        eq <= eq;
    end if;
end process;
\end{lstlisting}
\end{Code}

Этот код описывает схему с обратной связью, что приводит к образованию элемента памяти или внутреннему состоянию (обычно так называемых \emph{защелок} (\emph{latch})). Очевидно, что мы не это имели в виду. Код необходимо переписать следующим образом:

\begin{Code}
\begin{lstlisting}
process(a, b)
begin
    if (a = b) then
        eq <= '1';
    else
        eq <= '0';
    end if;
end process;
\end{lstlisting}
\end{Code}

Для описания комбинационной цепи необходимо всегда включать ветку \lstinline?else?, чтобы избежать образования элемента памяти. 

\subsubsection{Проблема неполного назначения сигналов в операторе IF}

В общем случае оператор \lstinline?if? имеет несколько веток. Возможно, что сигналу присваивается значение не во всех этих ветках. Хотя синтаксически правильно, это приводит к образованию элемента памяти в схеме (защелок). Пример компаратора двух чисел с тремя выходами: \lstinline?gt? (greater than, $a > b$), \lstinline?lt? (less than, $a < b$), \lstinline?eq? (equal, $a = b$).

\begin{Code}
\begin{lstlisting}
process(a, b)
begin
    if (a > b) then
        gt <= '1';
    elsif (a = b) then
        eq <= '1';
    else
        lt <= '1';
    end if;
end process;
\end{lstlisting}
\end{Code}

Опять же семантика VHDL подразумевает, что если сигналу не назначается значение, он его сохраняет. А это приводит к ненужному здесь образованию элемента памяти. Код следует исправить:

\begin{Code}
\begin{lstlisting}
process(a, b)
begin
    gt <= '0';
    eq <= '0';
    lt <= '0';
    if (a > b) then
        gt <= '1';
    elsif (a = b) then
        eq <= '1';
    else
        lt <= '1';
    end if;
end process;
\end{lstlisting}
\end{Code}

или:

\begin{Code}
\begin{lstlisting}
process(a, b)
begin
    if (a > b) then
        gt <= '1';
        eq <= '0';
        lt <= '0';
    elsif (a = b) then
        gt <= '0';
        eq <= '1';
        lt <= '0';
    else
        gt <= '0';
        eq <= '0';
        lt <= '1';
    end if;
end process;
\end{lstlisting}
\end{Code}

Первый вариант кода более красив и реализует прием программирования на VHDL, называемый присвоение значения комбинационному сигналу \emph{по умолчанию}. Здесь надо вспомнить, что при назначении сигнала в процессе, сигнал принимает последнее присвоенное значение. Рекомендуется использовать именно такой стиль программирования.

\subsection{Оператор CASE}

Оператор \lstinline?case? также является последовательным оператором, и его использование разрешено только внутри оператора процесса. Оператор \lstinline?case? имеет следующий синтаксис:

\begin{Code}
\begin{lstlisting}[mathescape]
case задающее$\_$выражение is
when выбор_1 =>
        последовательные$\_$операторы;
when выбор_2 =>
        последовательные$\_$операторы;
        ...
when выбор_n =>
        последовательные$\_$операторы;
end case;
\end{lstlisting}
\end{Code}

Выборочные значения \lstinline?выбор_i? должны быть полными и взаимоисключающими – как в операторе выборочного назначения сигнала (\lstinline?select?). Для обеспечения полноты выборочных значений рекомендуется использовать \lstinline?others?. Для компактного перечисления вариантов, которым соответствует один выходной код, можно использовать вертикальную черту \lstinline?"|"?. Рассмотрим на примерах.

\exsect{Пример 1.} Мультиплексор 4-в-1.

\begin{Code}
\begin{lstlisting}
architecture case_arch of mux4 is
begin
    process (a, b, c, d, s)
    begin
        case s is
            when "00" => x <= a;
            when "01" => x <= b;
            when "10" => x <= c;
            when others => x <= d;
        end case;
    end process;
end case_arch;
\end{lstlisting}
\end{Code}

\exsect{Пример 2.} Бинарный дешифратор.

\begin{Code}
\begin{lstlisting}
architecture case_arch of decoder4 is
begin
proc1: process (s)
    begin
        case s is
            when "00" => 
                x <= "0001";
            when "01" => 
                x <= "0010";
            when "10" => 
                x <= "0100";
            when others => 
                x <= "1000";
        end case;
    end process proc1;
end case_arch;
\end{lstlisting}
\end{Code}

Здесь \lstinline?proc1?~-- так называемая метка процесса. Язык VHDL допускает процессы без меток.

\exsect{Пример 3.} Шифратор приоритета.

\begin{Code}
\begin{lstlisting}
architecture case_arch of prio_encoder42 is
begin
process (r)
    begin
        case r is
            when "1000" | "1001" | "1010" | "1011" | 
                 "1100" | "1101" | "1110" | "1111" => 
                code <= "11";
            when "0100" | "0101" | "0110" | "0111" => 
                code <= "10";
            when "0010" | "0011" => 
                code <= "01";
            when others => 
                code <= "00";
        end case;
    end process;
    active <= r(3) or r(2) or r(1) or r(0);
end case_arch;
\end{lstlisting}
\end{Code}

В простых случаях оператор \lstinline?case? эквивалентен оператору выборочного назначения сигнала (\lstinline?select?):

\begin{Code}
\begin{lstlisting}[mathescape]
with задающее$\_$выражение select
    сигнал <=
        выражение_1 when выбор_1,
        выражение_2 when выбор_2,
        ...
        выражение_n when выбор_n;
\end{lstlisting}
\end{Code}

эквивалентно:

\begin{Code}
\begin{lstlisting}[mathescape]
process(...)
begin
    case задающее$\_$выражение is
        when выбор_1 =>
            сигнал <= выражение_1;
        when выбор_2 =>
            сигнал <= выражение_2;
            ...
        when выбор_n =>
            сигнал <= выражение_n;
    end case;
end process;
\end{lstlisting}
\end{Code}

\subsubsection{Проблема неполного назначения сигналов в операторе CASE}

Рассмотрим простой шифратор приоритета.

\begin{Code}
\begin{lstlisting}
process(a)
begin
    case a is
        when "100" | "101" | "110" | "111" =>
            high <= '1';
        when "010" | "011" =>
            middle <= '1';
        when others =>
            low <= '1';
    end case;
end process;
\end{lstlisting}
\end{Code}

Подразумевается, что в один момент времени только один из сигналов \lstinline?high?, \lstinline?middle? и \lstinline?low? будет равен \lstinline?'1'?. Однако при реализации схемы опять возникнет ненужный элемент памяти, так как не все сигналы назначаются за один "<пробег"> процесса. Решается эта проблема опять же присвоением значений сигналам по умолчанию.

\exsect{Упражнение.} Попробуйте написать соответствующий код самостоятельно и сравните с ответом.

\exsect{Ответ.}

\begin{Code}
\begin{lstlisting}
process(a)
begin
    high   <= '0';
    middle <= '0';
    low    <= '0';
    case a is
        when "100" | "101" | "110" | "111" =>
            high <= '1';
        when "010" | "011" =>
            middle <= '1';
        when others =>
            low <= '1';
    end case;
end process;
\end{lstlisting}
\end{Code}

\section{Правила описания комбинационных схем}
Перечислим правила, которых вы \textbf{должны} придерживаться при описании комбинационных схем с использованием оператора процесса и последовательных операторов:
\begin{itemize}
\item Все входные сигналы должны быть в списке чувствительности процесса.
\item Дерево условий оператора \lstinline?if? должно быть \textbf{полным}. Используйте \lstinline?else?.
\item Выборочные значения в операторе \lstinline?case? должны быть \textbf{полными и взаимоисключающими}. Используйте \lstinline?others?.
\item При выходе из процесса всем выходным сигналам необходимо присвоить значение (например, значение по умолчанию в начале процесса).
\end{itemize}

\section{Практика}
\subsection{Коммутатор-переключатель 2-на-2}
У коммутатора 2-на-2 есть два входа: \lstinline?x0? и \lstinline?x1? – и два выхода: \lstinline?y0? и \lstinline?y1?, а также вход управления \lstinline?ctrl?. В зависимости от значения сигнала \lstinline?ctrl? входные данные по-разному направляются на выходы.
\begin{table}[h]
\centering
\begin{tabular}{|c|c|c|l|}
\hline
input       & \multicolumn{2}{c|}{output} & function     \\ \hline
ctrl        & y1 & y0                    &              \\ \hline
\texttt{00} & x1 & x0                    & pass         \\
\texttt{01} & x0 & x1                    & cross        \\
\texttt{10} & x0 & x0                    & broadcast x0 \\
\texttt{11} & x1 & x1                    & broadcast x1 \\
\hline
\end{tabular}
\end{table}

\exsect{Упражнение.} Реализуйте схему переключателя на VHDL с использованием параллельных и последовательных операторов.

\subsection{Генератор следующего элемента последовательности Грея}
Код Грея~-- система счисления, в которой два соседних значения различаются только в одном разряде.

Изначально код Грея предназначался для защиты от ложного срабатывания электромеханических переключателей. Сегодня коды Грея широко используются для упрощения выявления и исправления ошибок в системах связи, а также в формировании сигналов обратной связи в системах управления. Коды Грея часто используются в датчиках-\emph{энкодерах} (датчиках угла поворота). Их использование удобно тем, что два соседних значения шкалы сигнала отличаются только в одном разряде. Также они используются для кодирования номера дорожек в жёстких дисках.

\begin{table}[h]
\centering
\begin{tabular}{|c|c|c|}
\hline
2-битный код Грея & 3-битный код Грея & 4-битный код Грея \\ \hline
\texttt{00}       & \texttt{000}      & \texttt{0000}     \\
\texttt{01}       & \texttt{001}      & \texttt{0001}     \\
\texttt{11}       & \texttt{011}      & \texttt{0011}     \\
\texttt{10}       & \texttt{010}      & \texttt{0010}     \\
                  & \texttt{110}      & \texttt{0110}     \\
                  & \texttt{111}      & \texttt{0111}     \\
                  & \texttt{101}      & \texttt{0101}     \\
                  & \texttt{100}      & \texttt{0100}     \\
                  &                   & \texttt{1100}     \\
                  &                   & \texttt{1101}     \\
                  &                   & \texttt{1111}     \\
                  &                   & \texttt{1110}     \\
                  &                   & \texttt{1010}     \\
                  &                   & \texttt{1011}     \\
                  &                   & \texttt{1001}     \\
                  &                   & \texttt{1000}     \\
\hline
\end{tabular}
\end{table}

Задача состоит в написании комбинационного модуля, генерирующего следующий элемент последовательности Грея, если на вход подан текущий элемент.

Самым простым и очевидным способом реализации модуля является простое задание таблицы соответствия с помощью операторов выборочного назначения сигналов.

\begin{Code}
\begin{lstlisting}
entity g_inc is
    port (
        g: in std_logic_vector(3 downto 0);
        g_next: out std_logic_vector(3 downto 0)
    );
end g_inc;

architecture table_arch of g_inc is
begin
    with g select
        g_next <= "0001" when "0000",
                  "0011" when "0001",
                  "0010" when "0011",
                  "0110" when "0010",
                  "0111" when "0110",
                  "0101" when "0111",
                  "0100" when "0101",
                  "1100" when "0100",
                  "1101" when "1100",
                  "1111" when "1101",
                  "1110" when "1111",
                  "1010" when "1110",
                  "1011" when "1010",
                  "1001" when "1011",
                  "1000" when "1001",
                  "0000" when others; -- "1000"
end table_arch;
\end{lstlisting}
\end{Code}

Подобный код не обладает свойством \emph{масштабируемости}. Он пригоден только для 4-битной шины. К сожалению, не существует простого алгоритма вычисления следующего элемента последовательности Грея непосредственно из предыдущего. Однако существуют несложные алгоритмы перевода кода Грея в двоичный код и обратно. Один из возможных подходов включает три этапа.

\begin{enumerate}
\item Преобразование кода Грея в соответствующий двоичный код.
\item Увеличение на единицу двоичного кода с использованием обычного сумматора.
\item Преобразование двоичного кода обратно в код Грея.
\end{enumerate}

\subsubsection{Преобразование двоичный код $\rightarrow$ код Грея}
Обозначим: $g$~-- число в представлении Грея, $b$~-- число в двоичном представлении, нижним индексом будем обозначать номер бита в представлении. Посмотрим внимательно на таблицу соответствия двоичных кодов и кодов Грея.

\begin{table}[h]
\centering
\begin{tabular}{|c|c|}
\hline
Двоичный код  & Код Грея      \\ \hline
\texttt{0000} & \texttt{0000} \\
\texttt{0001} & \texttt{0001} \\
\texttt{0010} & \texttt{0011} \\
\texttt{0011} & \texttt{0010} \\
\texttt{0100} & \texttt{0110} \\
\texttt{0101} & \texttt{0111} \\
\texttt{0110} & \texttt{0101} \\
\texttt{0111} & \texttt{0100} \\
\texttt{1000} & \texttt{1100} \\
\texttt{1001} & \texttt{1101} \\
\texttt{1010} & \texttt{1111} \\
\texttt{1011} & \texttt{1110} \\
\texttt{1100} & \texttt{1010} \\
\texttt{1101} & \texttt{1011} \\
\texttt{1110} & \texttt{1001} \\
\texttt{1111} & \texttt{1000} \\
\hline
\end{tabular}
\end{table}
 
Заметим, что $i$-й бит в представлении Грея (т.е. $g_i$) равен \texttt{'1'}, если $i$-й и $(i+1)$-й биты в соответствующем двоичном представлении (т.е. $b_i$ и $b_{i+1}$) различны, а старший бит всегда одинаков у обоих представлений. (Помним, что самый левый старший бит (Most Significant Bit, MSB) имеет самый большой номер, а самый правый младший бит (Least Significant Bit, LSB) имеет номер 0, т.е. нумерация ведется справа налево.) Для примера рассмотрим пятую строчку приведенной выше таблицы:

\begin{table}[h]
\centering
\begin{tabular}{|c|c|}
\hline
Двоичный код     & Код Грея         \\ \hline
\texttt{0100}    & \texttt{0110}    \\ \hline
\texttt{0\_\_\_} & \texttt{0\_\_\_} \\
\texttt{01\_\_}  & \texttt{\_1\_\_} \\
\texttt{\_10\_}  & \texttt{\_\_1\_} \\
\texttt{\_\_00}  & \texttt{\_\_\_0} \\
\hline
\end{tabular}
\end{table}

Это наблюдение запишем в виде логического уравнения с использованием оператора \emph{"<Исключающее ИЛИ">} или XOR:
\[
g_i = b_i \oplus b_{i+1}
\]

При этом для старшего бита следует вместо $b_{i+1}$  подставить \texttt{'0'}. Для 4-битного кода, таким образом, можно записать:
\[ \begin{aligned}
g_3 &= b_3 \oplus '0' = b_3 \\
g_2 &= b_2 \oplus b_3       \\
g_1 &= b_1 \oplus b_2       \\
g_0 &= b_0 \oplus b_1       \\
\end{aligned} \]

\subsubsection{Преобразование код Грея $\rightarrow$ двоичный код}

Заметим, что если $a \oplus b = c$, то $a = c \oplus b$.

\exsect{Упражнение.} Проверьте это утверждение, построив таблицу истинности.

Тогда из уравнения $g_i = b_i \oplus b_{i+1}$ следует, что
\[
b_i = g_i \oplus b_{i+1}
\]

Таким образом получена рекурсивная формула. Раскроем ее для случая 4-битных кодов.
\[ \begin{aligned}
b_3 &= g_3 \oplus '0' = g_3                                     \\
b_2 &= g_2 \oplus b_3 = g_2 \oplus g_3                          \\
b_1 &= g_1 \oplus b_2 = g_1 \oplus g_2 \oplus g_3               \\
b_0 &= g_0 \oplus b_1 = g_0 \oplus g_1 \oplus g_2 \oplus g_3    \\
\end{aligned} \]

\subsubsection{Реализация на VHDL}
\begin{Code}
\begin{lstlisting}
library ieee;
use ieee.std_logic_1164.all;
use ieee.std_logic_unsigned.all;

entity g_inc is
    generic (
        WIDTH: natural := 4
    );
    port ( 
        g : in std_logic_vector (WIDTH-1 downto 0);
        g_next : out std_logic_vector (WIDTH-1 downto 0)
    );
end g_inc;

architecture compact_arch of g_inc is
    signal b, b_next: std_logic_vector(WIDTH-1 downto 0);
begin
    -- Gray to binary
    b <= g xor ('0' & b(WIDTH-1 downto 1));
    -- binary increment
    b_next <= b + 1;
    -- binary to Gray
    g_next <= b_next xor ('0' & b_next(WIDTH-1 downto 1));
end compact_arch;
\end{lstlisting}
\end{Code}

\exsect{Упражнение.} Реализовать конвертер-генератор кода Грея. Входные сигналы: \lstinline?x?, \lstinline?s?. Сигнал \lstinline?x? имеет тип \lstinline?std_logic_vector(WIDTH-1 downto 0)?, а сигнал \lstinline?s? имеет тип \lstinline?std_logic_vector(1 downto 0)?.

\begin{table}[h]
\centering
\begin{tabular}{|c|c|}
\hline
input         & output                                                                                          \\ \hline
\lstinline?s? & \lstinline?y?                                                                                   \\ \hline
\texttt{00}   & \lstinline?y?~-- двоичное представление \lstinline?x?, \lstinline?x? записан в коде Грея        \\
\texttt{01}   & \lstinline?y?~-- представление \lstinline?x? в коде Грея, \lstinline?x? записан в двоичном коде \\
\texttt{10}   & \lstinline?y?~-- следующий элемент последовательности Грея, \lstinline?x?~-- текущий элемент    \\
\texttt{11}   & \lstinline?y?~-- предыдущий элемент последовательности Грея, \lstinline?x?~-- текущий элемент   \\
\hline
\end{tabular}
\end{table}

Использовать оператор процесса и последовательные операторы.

\section{Контрольные вопросы и задания}

\exsect{Теория}
\begin{enumerate}
\item Операторы условного и выборочного назначения сигналов в VHDL.
\item Оператор процесса и последовательные операторы (IF, CASE).
\item Оператор WAIT.
\item Правила описания комбинационных схем на VHDL.
\end{enumerate}

\exsect{Задания}
\begin{enumerate}
\item Разработайте комбинационный АЛУ, у которого на входе 2 шины операндов \lstinline?a? и \lstinline?b? (разрядность каждой~-- 8 бит) и шина выбора действий (разрядность~-- 2 бита). Выходная сигнал \lstinline?q? содержит 8 бит и равен сумме \lstinline?a? и \lstinline?b? при \lstinline?s = "00"?, разности~-- при \lstinline?s = "01"?, логическому ИЛИ – при \lstinline?s = "10"?, логическому И~-- при \lstinline?s = "11"?.
    \begin{enumerate}
    \item Используйте параллельные операторы.
    \item Используйте последовательные операторы.
    \end{enumerate}
\item Определим расстояние между элементами последовательности Грея. Пусть \lstinline?a? и \lstinline?b?~-- элементы последовательности Грея, тогда расстояние между \lstinline?a? и \lstinline?b? определим, как минимальное количество последовательных переходов от элемента \lstinline?a? до элемента \lstinline?b? по последовательности Грея. Например, пусть \lstinline?a = "0101"?, \lstinline?b = "1111"?. Тогда расстояние от \lstinline?a? до \lstinline?b? равно 4, поскольку требуется 4 перехода (см. таблицу с кодом Грея): \lstinline?"0101"? $\rightarrow$ \lstinline?"0100"? $\rightarrow$ \lstinline?"1100"? $\rightarrow$ \lstinline?"1101"? $\rightarrow$ \lstinline?"1111"?. Разработайте схему, вычисляющую расстояние между двумя 4-битными словами \lstinline?a? и \lstinline?b?, записанными в коде Грея.
\item \textbf{Задания на дополнительный код.} \emph{Дополнительный код} (англ. \emph{two’s complement})~-- наиболее распространённый способ представления отрицательных целых чисел в компьютерах. Он позволяет заменить операцию вычитания на операцию сложения и сделать операции сложения и вычитания одинаковыми для знаковых и беззнаковых чисел, чем упрощает архитектуру ЭВМ. Дополнительный код отрицательного числа можно получить инвертированием модуля двоичного числа и прибавлением к результату инверсии единицы. 
    \begin{enumerate}
    \item Разработайте комбинационный конвертер целых чисел со знаком. Схема имеет 2 входа: целое число со знаком \lstinline?a? (разрядность~-- 8 бит), сигнал выбора направления преобразования \lstinline?s? (разрядность~-- 1 бит). Если \lstinline?s = "0"?, то преобразуется целое число со знаком, записанное в дополнительном коде, в целое число со знаком, записанное в прямом коде, а если \lstinline?s = "1"?, то наоборот. Схема имеет единственный выход \lstinline?r? (разрядность~-- 8 бит).
    \item Используйте параллельные операторы.
    \item Используйте последовательные операторы.
    \end{enumerate}
\end{enumerate}
