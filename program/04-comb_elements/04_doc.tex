\chapter{Описание комбинационных устройств на VHDL}

\emph{Последовательные и параллельные операторы. Комбинационные логические элементы и их описание на языке VHDL. Операторы выборочного и условного назначения сигналов. Оператор процесса. Запуск процесса и список чувствительности. Операторы IF и CASE. Правила кодирования комбинационных логических устройств на VHDL.}

\section{Параллельное назначение сигналов в VHDL}

В общем случае цифровые устройства можно разделить на комбинационные и последовательные. Комбинационная цепь не имеет памяти, и значения на ее выходах в каждый момент времени зависят только от значений на ее входах.  Последовательная цепь, напротив, имеет память (или внутреннее состояние), и значения на ее выходах зависят как от входных сигналов, так и от внутреннего состояния, т.е. от предыстории функционирования цепи. Операторы параллельного назначения сигналов описывают комбинационные цепи.

\subsection{Простое назначение сигналов}
\begin{lstlisting}
status <= '1';
even <= (p1 and p2) or (p3 and p4);

arith_result <= a + b + c - 1;
\end{lstlisting}
Последнее выражение может быть изображено в виде следующей схемы:

\textbf{\textit{Упражнение.}} Предположим, что задержки сумматора и блока, вычисляющего разность входных сигналов, одинаковы и равны $\delta$, предположим, что других задержек нет (например, мы не учитываем задержки в проводниках хотя они порой очень существенны). Тогда изменение входного сигнала a отразится на выходном сигнале \texttt{arith\_result} только через $3 \delta$, ведь сигнал должен пройти через три блока, каждый из которых вносит задержку $\delta$. С другой стороны изменение входного сигнала c отразится на выходном сигнале \texttt{arith\_result} только через $2 \delta$. Т.е. схема еще и несимметричная. Можно ли оптимизировать схему, чтобы временная задержка стала меньше, и схема стала симметричной? (Ответ приведен далее, но все же подумайте, попробуйте переставить блоки.)

\textbf{\textit{Ответ.}} Воспользуемся коммутативностью и ассоциативностью сложения: 
\[
a + b + c - 1 = (a + b) + (c - 1).
\]
